Over the last decade, worst-case optimal join (\WCOJ) algorithms have
emerged as a new paradigm for one of the most fundamental challenges
in query processing: computing joins efficiently.  Such an algorithm
can be asymptotically faster than traditional binary joins, all the
while remaining simple to understand and implement.  However, they
have been found to be less efficient than the old paradigm,
traditional binary join plans, on the typical acyclic queries found in
practice.  Some database systems that support \WCOJ use a hybrid
approach: use \WCOJ to process the cyclic subparts of the query (if
any), and rely on traditional binary joins otherwise.  In this paper
we propose a new framework, called \FJ, that unifies the two
paradigms.  We describe a new type of plan, a new data structure
(which unifies the hash tables and tries used by the two paradigms),
and a suite of optimization techniques.  Our system, implemented in
Rust, matches or outperforms both traditional binary joins and \WCOJ on
standard query benchmarks.