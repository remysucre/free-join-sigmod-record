Over the last decade, worst-case optimal join (\WCOJ) algorithms have
emerged as a new paradigm for one of the most fundamental challenges
in query processing: computing joins efficiently.  Such an algorithm
can be asymptotically faster than traditional binary joins, all the
while remaining simple to understand and implement.  However, they
have been found to be less efficient than the old paradigm,
traditional binary join plans, on the typical acyclic queries found in
practice.
In an effort to unify and generalize the two paradigms,
we proposed a new framework, called \FJ, in our SIGMOD 2023 paper.
Not only does \FJ unite the worlds of traditional and worst-case optimal
join algorithms, it uncovers optimizations and evaluation strategies
that outperform both.

In this article, we approach \FJ from the traditional perspective
of binary joins, and re-derive the more general framework
via a series of gradual transformations.
We hope this perspective from the past can help practitioners
better understand the \FJ framework,
and find ways to incorporate some ideas into their own systems.

% Some database systems that support \WCOJ use a hybrid
% approach: use \WCOJ to process the cyclic subparts of the query (if
% any), and rely on traditional binary joins otherwise.  In this paper
% we propose a new framework, called \FJ, that unifies the two
% paradigms.  We describe a new type of plan, a new data structure
% (which unifies the hash tables and tries used by the two paradigms),
% and a suite of optimization techniques.  Our system, implemented in
% Rust, matches or outperforms both traditional binary joins and \WCOJ on
% standard query benchmarks.