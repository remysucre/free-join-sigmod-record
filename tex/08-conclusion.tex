\section{Conclusion}\label{sec:conclusion}
In this paper we review the \FJ framework,
which generalizes and unifies traditional join algorithms and \WCOJ algorithms.
We re-derive the more general \FJ from the well-understood binary join,
hoping to make the framework more accessible to database practitioners.
We hope to see the adoption of \FJ in mainstream databases,
which shall inspire further research on the design of join algorithms.
We conclude by pointing out some promising research directions.
First, we have been focusing on single-threaded in-memory algorithms.
How can we adapt \FJ to work on disk, on multi-core machines, and in distributed settings?
In particular, the COLT data structure relies on laziness and appears inherently sequential.
Do we need a new data structure to parallelize \FJ?
Second, our optimizer starts from an already optimized binary plan, and conservatively
improve it into a \FJ plan.
How can we design and implement an optimizer to better exploit the flexibilty of \FJ?
Finally, as the experiments show, our current implementation of \FJ is not yet competitive
for certain bushy plans.
How can we improve the performance of materializing intermediate results for \FJ?